\documentclass[11pt,a4paper]{article}
\usepackage[german,ngerman]{babel}%für richtige Datumsformattierung

\begin{document}
\selectlanguage{german}
\title{CS102 \LaTeX \hspace{1pt} \"Ubung}
\author{Ioana Mazare}
\maketitle

\section{Das ist der erste Abschnitt}
Hier k\"onnte auch ein anderer Text stehen.

\section{Tabelle}
Unsere wichtigsten Daten finden Sie in Tabelle 1.
\begin{table}[h]
\centering
\begin{tabular}{c|c|c|c}
 & Punkte erhalten & Punkte m\"oglich & \% \\
\hline
Aufgabe 1 & 2 & 4 & 0.5 \\
Aufgabe 2 & 3 & 3 & 1 \\
Aufgabe 3 & 3 & 3 & 1 \\
\end{tabular}
\caption{Diese Tabelle kann auch andere Werte beinhalten...}
\end{table}

\section{Formeln}
\subsection{Pythagoras}
Der Satz des Pythagoras errechnet sich wie folgt: $a^{2}+b^{2}=c^{2}$.
Daraus k\"onnen wir die L\"ange der Hypothenuse wie folgt berechnen: $c=\sqrt{a^{2}+b^{2}}$.

\subsection{Summen}
Wir k\"onnen auch die Formel f\"ur eine Summe angeben:
\begin{equation}
s=\sum_{i=1}^{n} i=\frac{n\times(n+1)}{2}
\end{equation}

\end{document}